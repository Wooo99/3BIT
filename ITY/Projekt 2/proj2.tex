\documentclass[a4paper, 11pt, twocolumn]{article}


\usepackage[czech]{babel}
\usepackage[utf8]{inputenc}
\usepackage[IL2]{fontenc}
\usepackage[left=1.5cm, top=2.5cm, text={18cm, 25cm}]{geometry}
\usepackage{amsthm}
\usepackage{amsmath}
\usepackage{verbatim}
\usepackage{mathtools}
\usepackage{amssymb}
\usepackage{amsfonts}
\usepackage{times}
\usepackage{titlesec}
\usepackage{setspace}


\newtheorem{definition}{Definice}

\newtheorem{sentence}{Věta}

\begin{document}
    \begin{titlepage}
		\begin{center}
			{\Huge\textsc{
			   	Vysoké učení technické v~Brně\\[0,4em]
            }}  
            {\huge\textsc{
                 Fakulta informačních technologií \\
			}}
			\vspace{\stretch{0.382}}
			{\LARGE{
				Typografie a~publikování \,--\,2. projekt\\[0,3em]
				Sazba dokumentů a~matematických výrazů
			}}
			\vspace{\stretch{0.618}}
		\end{center}

		{\Large
			\the\year
			\hfill
			Michal Řezník(xrezni28)
		}
	\end{titlepage}
	
	\section*{Úvod}
    	V~této úloze si vyzkoušíme sazbu titulní strany, matematic\-kých vzorců, prostředí a~dalších textových struktur obvyklých pro technicky zaměřené texty (například rovnice (\ref{eq_2}) nebo Definice \ref{def2}  na straně \pageref{def2}). Pro vytvoření těchto odkazů používáme příkazy \verb|\label|, \verb|\ref| a~\verb|\pageref|.
        
        Na titulní straně je využito sázení nadpisu podle optického středu s~využitím zlatého řezu. Tento postup byl probírán na přednášce. Dále je na titulní straně použito odřádkování se zadanou relativní velikostí 0,4~em a 0,3~em.

    \section{Matematický text}
        Nejprve se podíváme na sázení matematických symbolů a~výrazů v~plynulém textu včetně sazby definic a~vět s~využitím balíku \texttt{amsthm}. Rovněž použijeme poznámku pod čarou s~použitím příkazu \verb|\footnote|. Někdy je vhodné použít konstrukci \verb|${}$| nebo \verb|\mbox{}|, která říká, že (matematický) text nemá být zalomen. 
        
        \begin{definition}
        Nedeterministický Turingův stroj (NTS)\emph{je šestice tvaru} $ M = (Q,\Sigma, \Gamma, \delta, q_0, q_F) $, kde:
        \begin{itemize}
            \item $ Q $ je konečná množina \emph{vnitřních (řídicích) stavů},
            \item $ \Sigma $ je konečná množina symbolů nazývaná \emph{vstupní abeceda,} $ \Delta \notin \Sigma$,
            \item $ \Gamma $ je konečná množina symbolů, $\Sigma \subset \Gamma, \Delta \in \Gamma $, nazývaná \emph{pásková abeceda,}
            \item $ \delta :(Q \backslash \{ $$q_F \}) \times \Gamma \rightarrow $$2^{Q\times(\Gamma\cup\{L,R\})}$\emph{, kde $L, R \notin \Gamma,$ je parciální přechodová funkce, a}
            \item $ q_0 \in Q$ je \emph{počáteční stav} a $ q_F \in Q$ je \emph{koncový stav.}
        \end{itemize}
        
        \end{definition}
        Symbol $ \Delta $ značí tzv. \emph{blank} (prázdný symbol), který se vyskytuje na místech pásky, která nebyla ještě použita.
         
        \textit{Konfigurace pásky} se skládá z~nekonečného řetězce, který reprezentuje obsah pásky, a pozice hlavy na tomto řetězci. Jedná se o~prvek množiny $ \{\gamma\Delta^\omega \mid \gamma \in. \Gamma^*\} \times \mathbb{N}$\footnote{Pro libovolnou abecedu $ \Sigma $ je $ \Sigma^\omega $ množina všech nekonečných řetězců nad $ \Sigma $, tj. nekonečných posloupností symbolů ze $ \Sigma $}
        \textit{Konfiguraci pásky} obvykle zapisujeme jako $ \Delta xyzzx\Delta $\dots\\ (podtržení značí pozici hlavy).
        \textit{Konfigurace stroje} je pak dána stavem řízení a~konfigurací pásky. Formálně se jedná o~prvek množiny $ Q \times \{\gamma\Delta^\omega \mid \gamma \in. \Gamma^*\} \times \mathbb{N} $.
        
        \subsection{Podsekce obsahující defnici a~větu}
        \begin{definition}
        \label{def2}
        \emph{Řetězec $ w $ nad abecedou $ \Sigma $ je přijat NTS $ M $,}\\ jestliže $ M $ při aktivaci z počáteční konfigurace pásky $ \underline{\Delta}w\Delta $\dots a~počátečního stavu $ q_0 $ může zastavit přechodem do koncového stavu $ q_F $, tj. $ (q_0,\Delta w\Delta^w,0) \overset{*}{\underset{M}{\vdash}} (q_F,\gamma,n)$  pro nějaké $ \gamma \in \Gamma^* $ a $ n \in \mathbb{N} $.
        
        Množinu $ L(M) = \{w \mid w $ je přijat NTS $ M\} \subseteq \Sigma^* $ nazýváme \emph{jazyk přijímaný NTS} $ M $.
        \end{definition}
        Nyní si vyzkoušíme sazbu vět a~důkazů opět s~použitím balíku \texttt{amsthm}.
        \begin{sentence}
        Třída jazyků, které jsou přijímány NTS, odpovídá \emph{rekurzivně vyčíslitelným jazykům.}
        \end{sentence}
    \section{Rovnice}
        Složitější matematické formulace sázíme mimo plynulý text. Lze umístit několik výrazů na jeden řádek, ale pak je třeba tyto vhodně oddělit, například příkazem \verb|\quad|.
        
        $$
        x^2 - \sqrt[4]{y_1*y^3_2}
        \quad
        x > y_1 \geq y_2
        \quad
        z_{z_z}\neq \alpha_1^{\alpha_2^{\alpha_3}}
        $$
        
        V rovnici (\ref{eq_1}) jsou využity tři typy závorek s~různou explicitně definovanou velikostí.
        
        \begin{eqnarray}
        \label{eq_1} x & = & \bigg\{a \oplus\Big[b \cdot\big(c\ominus d\big)\Big]\bigg\}^{4/2}\\
        \label{eq_2} y & = & \lim_{\beta\to\infty} \frac{\tan^2\beta - \sin^3\beta}{\frac{1}{\frac{1}{\log_{42}{x}}+\frac{1}{2}}}
        \end{eqnarray}
        
        V této větě vidíme, jak vypadá implicitní vysázení limity $ \lim_{n\to\infty} f(n) $ v~normálním odstavci textu. Podobně je to i~s~dalšími symboly jako $ \bigcup_{N \in \mathcal{M}} N $ či $ \sum^n_{j=0}x^2_j $.
        S vynucením méně úsporné sazby příkazem \verb|\limits| budou vzorce vysázeny v podobě $ \lim\limits_{n\to\infty} f(n)$ a~$\sum\limits_{j=0}^{n}x^2_j $.
        
        \section{Matice}
        Pro sázení matic se velmi často používá prostředí \texttt{array} a~závorky (\verb|\left|, \verb|\right|).
        
        $$
		\mathbf{A} =
		\left|
		\begin{array}{cccc}
			a_{11} & a_{12} & \ldots & a_{1n} \\
			a_{21} & a_{22} & \ldots & a_{2n} \\
			\vdots & \vdots & \ddots & \vdots \\
			a_{m1} & a_{m2} & \ldots & a_{mn}
		\end{array}
		\right|
		=
		\left|
		\begin{array}{cc}
			t & u \\
			v & w
		\end{array}
		\right|
		= tw - uv
	    $$
	    
	    Prostředí \texttt{array} lze úspěšně využít i~jinde.
	    
	    $$
		\binom{n}{k} =
		\left\{
		\begin{array}{ll}
			\frac{n!}{k! (n - k)!} & \text{pro } 0 \leq k \leq n \\
			 \quad 0 & \text{pro } k > n \text{ nebo } k < n\\
		\end{array}
		\right.
    	$$

\end{document}