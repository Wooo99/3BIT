\documentclass[11pt, a4paper]{article}

\usepackage{times}
\usepackage[T1]{fontenc}
\usepackage[left=2cm, text={17cm, 24cm}, top=3cm]{geometry}
\usepackage[czech]{babel}
\usepackage[utf8]{inputenc}


\begin{document}
\begin{titlepage}

\begin{center}
	\Huge
 	\textsc{Vysoké učení technické v~Brně\\
 		{\huge Fakulta informačních technologií}}\\
 		
	\vspace{\stretch{0.382}}
	\LARGE
	 Typografie a publikování - 4. projekt\\ 
	 	\huge {Bibliografické citace}
	\vspace{\stretch{0.618}}
\end{center}

{\Large \today \hfill Michal Řezník}
\end{titlepage}
\section*{Úvod do typografie}
    Je tomu už několik tisíc let, kdy se začalo poprvé používat písmo. Od kněžích a důležitých osob se po určité době rozšířilo mezi veškeré obyvatelstvo. V dnešní době se již můžeme setkat s nejrůznějšími pravidly, které správná typografie vyžaduje. Správná úprava knih, textů nebo stránek vede k daleko lešpímu zážitku a dojmu z nich viz. \cite{Noga2010}. Jako průvodce grafickou úpravou, včetně typografie můžeme použít například knihu \cite{Harris2009}.

\subsection*{Grafická úprava}
    \emph{Typografie je jednou z nejdůležitějších součástí grafického designu.}Touto větou začal článek \cite{Ovsyannykov2021}, který popisuje začátky práce s typgorafií. Její důležitost zpracoval i Bc. Matěj Kašpar Jirásek ve své bakalářské praci \cite{Jirasek2016} zaměřené na srovnání typografie webu a tištěných médií.

\section*{Využívání {\LaTeX}u}
    \LaTeX je typografický systém, který je určen jak k sazbě vědeckých, tak i různých kretivně zaměřených dokumentů. 
    Tento systém je postaven na sázecím programu Tex viz \cite{Olsak2021}.
    Podrobnější informace se můžeme dočíst v článku \cite{Davidek2019}. Latex má nespočet návodů a knih, které mají sloužit jako průvodce do začátků viz. \cite{Rybicka2003}.
    
    Jako největší výhodu {\LaTeX}u bych popsal jeho jednoduchost v matematických vzorcích a technologických zápisech. Kde konvenční editory, například od společnosti Microsoft, zaostávají.
    
    K použití Latexu může sloužit nespočet editorů, jako například intnernetové, které často velmi zjednodušují práci a využívání {\LaTeX}u. 
    
\section*{Typografie jako vědní obor}
    Typografie je mimo disciplínu zabývající se písmen také obor, který může sloužit při určování autora textů viz \cite{Beranek2021}. Jinou zkoumanou oblastí je psychologický pohled, kdy změna fontu nebo rozložení stránky dokáže, například u internetových obchodů, zvýšit počet objednávek. Psychologií fontů se například zabývá článek \cite{Grigerova2019}.
    
\section*{Závěr}    
    Typografie má v České republice velké zastoupení, asi největsí skupinou je Československé sdružení uživatelů TeXu, jež pravidelně vydávali časopis věnovaný právě TeXu. Dále například vychází časopis zaměřený výhradně na typografii \cite{FONT}.
    
\newpage
	\bibliographystyle{czechiso}
	\renewcommand{\refname}{Literatura}
	\bibliography{proj4}




\end{document}
